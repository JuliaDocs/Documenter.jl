% Useful variables
\newcommand{\DocMainTitle}{Documenter LaTeX Simple Non-Docker}
\newcommand{\DocVersion}{1.2.3}
\newcommand{\DocAuthors}{}
\newcommand{\JuliaVersion}{1.8.0}

% ---- Insert preamble
%% Default preamble BEGIN
\documentclass[oneside]{memoir}

\usepackage{./documenter}
\usepackage{./custom}

%% Title Page
\title{
    {\HUGE \DocMainTitle }\\
    {\Large \DocVersion }
}
\author{ \DocAuthors }

%% TOC settings
% -- TOC depth
%   value: [part, chapter, section, subsection,
%           subsubsection, paragraph, subparagraph]
\settocdepth{section}  % show "part+chapter+section" in TOC
% -- TOC spacing
%   ref: https://tex.stackexchange.com/questions/60317/toc-spacing-in-memoir
%   doc: memoir/memman.pdf
%       - Figure 9.2: Layout of a ToC
%       - Table 9.3: Value of K in macros for styling entries
\makeatletter
% {part} to {chaper}
\setlength{\cftbeforepartskip}{1.5em \@plus \p@}
% {chaper} to {chaper}
\setlength{\cftbeforechapterskip}{0.0em \@plus \p@}
% Chapter num to chapter title spacing (Figure 9.2@memman)
\setlength{\cftchapternumwidth}{2.5em \@plus \p@}
% indent before section number
\setlength{\cftsectionindent}{2.5em \@plus \p@}
% Section num to section title spacing (Figure 9.2@memman)
\setlength{\cftsectionnumwidth}{4.0em \@plus \p@}
\makeatother

%% Main document begin
\begin{document}

\frontmatter
\maketitle
\clearpage
\tableofcontents

\mainmatter
%% preamble END



\part{Main section}


\chapter{Simple LaTeX build}



\label{12986824182765986604}{}


This build only contains a single paragraph of text to make sure that a near-empty LaTeX builds passes.




\begin{minted}{jlcon}
julia> 127 % Int8
127
\end{minted}



\section{Issue 1119}



\label{496847156375218796}{}



\begin{minted}[escapeinside=\#\%]{julia}
1 #%% 2
1 #\unicodeveebar% 2
1 | 2
\end{minted}



\section{Escaping: {\textasciitilde}, {\textasciicircum}, {\textbackslash}, {\textquotesingle}, {\textquotedbl}, \_, \&, \%, {\textbackslash}, \$, \#, \{ and \}.}



\label{13534704973354818981}{}


{\textasciitilde}, {\textasciicircum}, {\textbackslash}, {\textquotesingle}, {\textquotedbl}, \_, \&, \%, {\textbackslash}, \$, \#, \{ and \}.



\begin{quote}
\textbf{{\textasciitilde}, {\textasciicircum}, {\textbackslash}, {\textquotesingle}, {\textquotedbl}, \_, \&, \%, {\textbackslash}, \$, \#, \{ and \}.}

{\textasciitilde}, {\textasciicircum}, {\textbackslash}, {\textquotesingle}, {\textquotedbl}, \_, \&, \%, {\textbackslash}, \$, \#, \{ and \}.

\end{quote}


\section{Issue 1392}



\label{987107409634109244}{}



\begin{minted}{jlcon}
julia> function foo end;
\end{minted}



\section{Issue 1401: rendering custom LaTeX}



\label{1766923436598802013}{}


\subsection{@example block / show with \texttt{text/latex} MIME}



\label{12212591717920375664}{}



\begin{minted}{julia}
struct Table end

Base.show(io, ::MIME"text/latex", ::Table) = write(io, raw"""
\begin{tabular}{r|ccc}
        & i & y & z\\
        \hline
        & Int64 & Char & Int64\\
        \hline
        1 & 1 & 'A' & 5 \\
        2 & 2 & 'B' & 6 \\
        3 & 3 & 'C' & 7 \\
        4 & 4 & 'D' & 8 \\
\end{tabular}
""")

Table()
\end{minted}

\begin{tabular}{r|ccc}
        & i & y & z\\
        \hline
        & Int64 & Char & Int64\\
        \hline
        1 & 1 & 'A' & 5 \\
        2 & 2 & 'B' & 6 \\
        3 & 3 & 'C' & 7 \\
        4 & 4 & 'D' & 8 \\
\end{tabular}


\subsection{@raw block}



\label{12968435574917214693}{}



\begin{tabular}{r|ccc}
        & i & y & z\\
        \hline
        & Int64 & Char & Int64\\
        \hline
        1 & 1 & 'A' & 5 \\
        2 & 2 & 'B' & 6 \\
        3 & 3 & 'C' & 7 \\
        4 & 4 & 'D' & 8 \\
\end{tabular}



\subsection{Inline LaTeX}



\label{7647595013011656041}{}


\emph{Note: this should render as just text, not as a table.}



{\textbackslash}begin\{tabular\}\{r|ccc\}         \& i \& y \& z{\textbackslash}
        {\textbackslash}hline         \& Int64 \& Char \& Int64{\textbackslash}
        {\textbackslash}hline         1 \& 1 \& {\textquotesingle}A{\textquotesingle} \& 5 {\textbackslash}
        2 \& 2 \& {\textquotesingle}B{\textquotesingle} \& 6 {\textbackslash}
        3 \& 3 \& {\textquotesingle}C{\textquotesingle} \& 7 {\textbackslash}
        4 \& 4 \& {\textquotesingle}D{\textquotesingle} \& 8 {\textbackslash}
{\textbackslash}end\{tabular\}



\section{Printing LaTeX from Julia}



\label{10914504187442884412}{}


To pretty-print LaTeX from Julia, overload \texttt{Base.show} for the \texttt{MIME{\textquotedbl}text/latex{\textquotedbl}} type. For example:




\begin{minted}{julia}
struct LaTeXEquation
    content::String
end

function Base.show(io::IO, ::MIME"text/latex", x::LaTeXEquation)
    # Wrap in $$ for display math printing
    return print(io, "\$\$ " * x.content * " \$\$")
end

LaTeXEquation(raw"""
    \left[\begin{array}{c}
        x \\
        y
    \end{array}\right]
""")
\end{minted}

$$     \left[\begin{array}{c}
        x \\
        y
    \end{array}\right]
 $$


\begin{minted}{julia}
struct LaTeXEquation2
    content::String
end

function Base.show(io::IO, ::MIME"text/latex", x::LaTeXEquation2)
    # Wrap in \[...\] for display math printing
    return print(io, "\\[ " * x.content * " \\]")
end

LaTeXEquation2(raw"""
    \left[\begin{array}{c}
        x \\
        y
    \end{array}\right]
""")
\end{minted}

\[     \left[\begin{array}{c}
        x \\
        y
    \end{array}\right]
 \]

\section{\texttt{LineBreak} node}



\label{18112929726868774212}{}


This sentence\\
should be over \textbf{multiple\\
lines}.



\end{document}
